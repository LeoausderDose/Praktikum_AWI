%Matteo Kumar - Leonard Schatt
% Fortgeschrittenes Physikalisches Praktikum
% Main-Datei für die Auswertung in TeX

% Struktur:
% Für jeden Abschnitt gibt es einen Ordner, damit jeder individuell an seinen Aufgaben arbeiten
% kann, ohne beim merge in GitHub Konflikte zu erhalten. Deshalb werden alle Unteraufgaben auch 
% extra in Ordner angelegt. Die einzelnen Dateien über den input Befehl einfügbar.
% Bilder und andere Grafik werden im Ordner Grafik abgelegt 


% Packages
\documentclass[paper=a4,bibliography=totoc,BCOR=10mm,twoside,numbers=noenddot,fontsize=11pt]{scrreprt}
\usepackage[english]{babel}
\usepackage[latin1, utf8]{inputenc}
\usepackage[babel]{csquotes} %For Quotes
\usepackage[T1]{fontenc}
\usepackage{lmodern}
\usepackage{graphicx}
\usepackage{nicefrac}
\usepackage{fancyvrb}
\usepackage{amsmath,amssymb,amstext}
\usepackage{siunitx}
\usepackage{physics}


%\usepackage{url}  % Wird schon in hyperref importiert
\usepackage{natbib}
\usepackage{microtype}
\usepackage[format=plain]{caption}
\usepackage{titleref}

% Zusätzliche Packages
\usepackage{geometry}
\usepackage{anyfontsize}
\usepackage[table]{xcolor}
\usepackage{ifthen}
\usepackage[absolute,overlay]{textpos}
\usepackage{amsfonts}
\usepackage{xstring}
%\usepackage{tikz}
%\usepackage{pdfpages}
\usepackage{booktabs}
\usepackage{hyperref}
\usepackage[capitalize]{cleveref}
\usepackage{subcaption}



% Abschnittseinrückung und -abstand
% Die folgenden Zeilen sollen möglichst nicht verändert werden
\parindent 0.0cm
\parskip 0.8ex plus 0.5ex minus 0.5ex

% Anzahl und Größe von Gleitobjekten
% maximal 2 Objekte oben und unten
% erlaubt auch größere Bilder, welche die ganze Seite benötigen
% Die folgenden Zeilen sollen möglichst nicht verändert werden
\setcounter{bottomnumber}{2}
\setcounter{topnumber}{2}
\renewcommand{\bottomfraction}{1.}
\renewcommand{\topfraction}{1.}
\renewcommand{\textfraction}{0.}
\AtBeginDocument{\RenewCommandCopy\qty\SI}


%\sc und \bc veraltet. Daher: (20.09.2018)
\DeclareOldFontCommand{\sc}{\normalfont\scshape}{\@nomath\sc}
\DeclareOldFontCommand{\bf}{\normalfont\scshape}{\textbf}

% Verschiedenes
\pagestyle{headings}          % Der Seitenstil sollte möglichst nicht verändert werden
\graphicspath{{./bilder/}}    % Der Pfad für die Abbildungen Abbildungen wird gesetzt
\VerbatimFootnotes            % \verb etc.

% Funktionen
\newcommand\tab[1][1cm]{\hspace*{#1}}
\newcommand{\vect}[1]{\boldsymbol{\mathbf{#1}}}
\newcolumntype{g}{>{\columncolor[rgb]{ .741,  .843,  .933}}l}

%Setup Caption etc.
\captionsetup[figure]{labelfont={bf}}



\begin{document}

    \nonfrenchspacing

    % 0. Kapitel 
    \begin{titlepage}
    \centering
    \vspace*{5cm}
    {\Huge \textbf{Internship AWI} \\[3ex]}
    \newpage
\end{titlepage}

\newpage

\restoregeometry



    \thispagestyle{empty}
    \cleardoublepage
    \tableofcontents
    \cleardoublepage

    % 1. Kapitel Einleitung
    %Matteo Kumar - Leonard Schatt
% Fortgeschrittenes Physikalisches Praktikum

% 1. Kapitel Einleitung

\chapter{Introduction}
\label{chap:einleitung}
 
The diurnal temperature range (DTR), the difference between daily minimum temperatures ($T_{\mathrm{min}}$) and daily maximum temperatures ($T_{\mathrm{max}}$), decreases on a global scale and in many regional scales since the 1990s \cite{Easterling.1997,Karl.1993,Alexander.2006}. This decrease is caused by the mean surface minimum temperatures rising faster than the maximum temperatures \cite{Karl.1993}. The reasons for this dynamic are quantitatively not fully understood yet, but the asymmetry in warming has been linked to factors of external forcing such as greenhouse gases, aerosols, and changes in landscape due to land use, as well as secondary effects such as changes in cloudiness and relative humidity \cite[text]{Zhou.2010,Lewis.2013,Liu.2016,Christidis.2016}. The extent of change in DTR is of magnitudes smaller than the trends in mean surface temperatures ($T_{\mathrm{mean}}$). The uncertainty in these trends surpasses that of surface mean temperature, and they were only assessed with medium confidence in the fourth and fifth IPCC reports \cite{.2014}.

In this work, we examine the DTR for the Arctic region. The mean surface temperatures in the Arctic warm more rapidly than in any other region on Earth due to Arctic amplification \cite{Pithan.2014}. At the same time, the Arctic shows great climate variability, which decreases the signal-to-noise ratio (SNR) of the trends in mean surface temperature\cite{Hartmuth.2023}. As there is no linear coupling between the diurnal temperature range and the mean temperature, the DTR might be less influenced by the high climate variability. In the following, we examine whether the trends in the diurnal temperature range show a better SNR in the Arctic region than the mean surface temperature. This could lead to a more robust parameter to track changes in the polar regions.

In section (sec1), we investigate factors causing the trends in the Diurnal Temperature Range (DTR) at the AWIPEV meteorological station in Spitsbergen. Since the average temperatures in Antarctica have not changed despite climate change, in section (ref), we address the question of whether there have been shifts in the daily temperature range.
% Studies have show a decrease of the diurnal temperature range (DTR), the difference between daily minimum temperatures ($T_{\mathrm{min}}$) and daily maximum temperatures ($T_{\mathrm{max}}$), on a global scale and in many regions in regional scale. This decrease is caused by the
% mean surface minimum temperatures rising faster then the maximum temperatures, since the 1990s. The reasons for this 
% dynamic is quantitatively not fully understood yet, but the asymmetry in warming was linked to factors of 
% external forcing such as greenhouse gases, aerosols and change of landscape by landuse and secondary effects such as changes in cloudiness and relative humidity. 
% The extend of change in DTR are of magnitudes smaller than the trends in mean surface temperatures ($T_{\mathrm{mean}}$).
% The uncertainty in these trends surpasses that of surface mean temperature, and they were only assessed with medium confidence in the fourth and fifth IPCC reports.



% In this work, we examine the DTR for the arctic region.
% The mean surface temperatures in arctic warms change more rapid than in any other region on earth due to arctic amplification. At the
% same time, the arctic shows great climate variability which decreases the signal-to-noise ratio (SNR) of the trends in mean surface temperature.
% As there is no linear coupling between the diurnal temperature range and the mean temperature, the DTR might be less influenced by the high climate variability.
% In the following, we examine whether the trends in the diurnal temperature range show a better SNR in the 
% arctic region than the mean surface temperature. This could lead to more robust parameter to track changes in the polar regions. 
% % In the sections (sec1), the causes of the change in DTR at the meterological station AWIPEV are more closely investigated.
% % As there are no observable changes due to climate change in mean temperature yet, we investigate in sec (ref), whether the are already changes 
% % in the diurnal temperature range visible. 
% In section (sec1), we investigate factors causing the trends in the Diurnal Temperature Range (DTR) at the AWIPEV meteorological station in Spitzbergen. 
% Since the average temperatures in Antarctica have not changed despite climate change, in section (ref) we address the question of whether there have been shifts in the daily temperature range.


    % 2.Kapitel Fragen zur Vorbereitung
    %Matteo Kumar - Leonard Schatt
% Fortgeschrittenes Physikalisches Praktikum
 
\chapter{Data and methods}
\label{cha:theory}

% \section{Data}
The diurnal temperature range (DTR) data, analyzed at a multi-regional scale in the following section, is derived from the \textit{CRU TS v4.07} dataset \cite{Harris.2020}. This dataset offers a spatial resolution of $\SI{0.5}{\degree} \times \SI{0.5}{\degree}$ and a monthly temporal resolution dating back to 1901.
The diurnal temperature range is defined as the difference
\begin{equation}
    DTR = T_{\mathrm{max}} - T_{\mathrm{min}}
\end{equation}
of the $T_{\mathrm{max}}$ and $T_{\mathrm{min}}$ representing the maximum and minimum temperatures of the day. In spatial averaging we weight by area.

The local data for the Arctic were taken from 'AWIPEV,' the station for continuous meteorological observations in Ny-Ålesund \cite{Maturilli.2020}. For Antarctica, we use the time series data from the 'Georg von Neumayer' station \cite{KonigLanglo.2017}. When multiple datasets for the same quantity exist at the same time, the 
quantities are averaged. 



    % 3.Kapitel Protokoll
    %
% Physikalisches Praktikum

% 3.Kapitel  Protokoll

% Variables
\def\skalierung{0.65}

\chapter{Materials, Methods and Setup}
\label{chap:methods}

\section{Dark-field microscopy}
\label{sec:DarkFMicro}

The following section explains the set-up for dark field microscopy and the process of adjustment. 

\subsection{Setup}

The setup itself consist of a system of lenses, a light source one mask and one camera as schematically depicted in \cref{fig:DarkFMicro}. 
In our case all components were realized in a plug system of ThorLabs. We used a white LED as light source and a weak LASER to improve our adjustment of the setup. 
The setup is briefly introduced by following the path of the light. 

\begin{figure}[ht]
    \centering
    \includegraphics[width = \linewidth]{Bilder/Setup/MikroskopEdit.png}
    \caption{Schematic representation of the used dark-field microscope. From \cite{LehrstuhlExperimentalphysikIII.2023}}
    \label{fig:DarkFMicro}
\end{figure}

The first lense parallelizes the light. After this a dark field mask (DF mask) blocks the light which would directly enter in teh second lense of the microscope. The light which is transmitted by the DF mask is guided onto the sample and 
is scattered by the structure. This scattered light is picked up by the lense after the sample and converted to an image using the rest of the lenses. This picture is captured by the camera.

\subsection{Adjusting the setup}

The setup adjustment consists of multiple steps. Firstly, the positioning of the first lens is adjusted. Next, the second lens is focused on the sample, followed by guiding the beam into the spectrometer using two mirrors.

To adjust the position of the first lens, we incorporate a beam splitter in the beam path before the lens. The lens is then focused on the sample. To verify the focus we aim for a sharp image of the reflection, which can be observed through the beam splitter, as shown in Figure \ref{fig:subfig1}.
For this the position of the sample has to be changed using the millimeter screws until we see a the nano particles.


\begin{figure}[ht]
    \centering
    \begin{subfigure}{0.3\linewidth}
      \includegraphics[width=\linewidth]{data/Gruppe2/image_0.png}
      \caption{}
      \label{fig:subfig1}
    \end{subfigure}
    \begin{subfigure}{0.3\linewidth}
      \includegraphics[width=\linewidth]{data/Gruppe2/image_1.png}
      \caption{}
      \label{fig:subfig2}
    \end{subfigure}
    \begin{subfigure}{0.3\linewidth}
      \includegraphics[width=\linewidth]{data/Gruppe2/image_2.png}
      \caption{}
      \label{fig:subfig3}
    \end{subfigure}

    \begin{subfigure}{0.3\linewidth}
      \includegraphics[width=\linewidth]{data/Gruppe2/image_3.png}
      \caption{}
      \label{fig:subfig4}
    \end{subfigure}
    \begin{subfigure}{0.3\linewidth}
      \includegraphics[width=\linewidth]{data/Gruppe2/image_4.png}
      \caption{}
      \label{fig:subfig5}
    \end{subfigure}
    \begin{subfigure}{0.3\linewidth}
      \includegraphics[width=\linewidth]{data/Gruppe2/image_5.png}
      \caption{}
      \label{fig:subfig6}
    \end{subfigure}


    \begin{subfigure}{0.3\linewidth}
      \includegraphics[width=\linewidth]{data/Gruppe2/image_6.png}
      \caption{Horizontal polarizations}
      \label{fig:subfig7}
    \end{subfigure}
    \begin{subfigure}{0.3\linewidth}
      \includegraphics[width=\linewidth]{data/Gruppe2/image_7.png}
      \caption{Vertical polarizations}
      \label{fig:subfig8}
    \end{subfigure}
    \begin{subfigure}{0.3\linewidth}
      \includegraphics[width=\linewidth]{data/Gruppe2/image_8.png}
      \caption{}
      \label{fig:subfig9}
    \end{subfigure}
    
    \caption{The pictures were generated during adjustment of the setup. (a) shows the reflection of the film captured with a camera positioned at the place of the beam splitter. The pictures (b)-(i) ones were taken in transmission, after the adjustment of the second lens.}
    \label{fig:subfigure-grid}
\end{figure}

As a second step, the position of the second lens is optimized using a LASER. The LASER is quickly mounted instead of the LED. The diameter of the LASER spot is minimized by changing the distance of the second lens.

After inserting a second beam splitter following the second lens, we are able to view the transmission picture using a camera, as shown in \cref{fig:subfig2,fig:subfig3,fig:subfig4,fig:subfig5,fig:subfig6,fig:subfig7,fig:subfig8,fig:subfig9}. The contrast increasing effects
of the dark-field microscopy can be easily seen in \cref{fig:subfigure-grid} after placing the DF mask. The positioning of the mask gives artefacts as can be seen in comparing \cref{fig:subfig2,fig:subfig9}. The sample is examined to determine its position by identifying the markers present non its edges. In \cref{fig:subfig7,fig:subfig8,fig:subfig9}, the marker with text can be seen in the bottom left corner. 
By comparing this to the plan for the sample depicted in \cref{fig:NanoDotSketch} and taking into account the orientation of the 
letters on the marker, we can tell that the sample is orientated as on the plan. When inserting polarizations filters as done in \cref{sub@fig:subfig7,sub@fig:subfig9}, we can not notice a difference in the scattering without numerical analysis. A slight shift in color might be visible.  
 

\begin{figure}[ht]
    \centering
    \includegraphics[width = 0.8\linewidth]{Bilder/Setup/SchemeDots.png}
    \caption{Schematic representation of the nano dot structure used. From \cite{LehrstuhlExperimentalphysikIII.2023}}
    \label{fig:NanoDotSketch}
\end{figure}

Afterwards, the beam is directed into the spectrometer by utilizing two mirrors. The slit is fully opened. Using the spectrometer, we examine a small image captured through the slit, which allows us to determine our location on the sample by identifying distinctive contaminants present. Once the sample position is determined, we reduce noise by closing the slit and manually positioning the sample using the second beam splitter and the camera to orient ourselves on the sample. When it comes to measurement, we remove both beam splitters.
\section{Sample}
\label{sec:sample}

The sample used is a sample of nanorods depicted in \ref{fig:NanoDotSketch}. 
The sample was created through electron beam lithography. It is placed on a cover glass (BK7) with an estimated thickness of around \SI{170}{\micro\meter}. The structure stands at a height of \SI{35}{\nano\meter} and is composed of silver, where in contrast to gold, all electrons can be approximated as free electrons. Due to the writing and deposition process, the structure is planar and consists of nanorods with widths of \SI{70}{\nano\meter} and \SI{90}{\nano\meter} and variable length L. The length of the nanorods, denoted as L, is varied.

    % 4.Kapitel Versuchsauswertung
    % Matteo Kumar - Leonard Schatt
% Fortgeschrittenes Physikalisches Praktikum
% 4.Kapitel Versuchsauswertung

\chapter{Discussion}
\label{chap:Discussion}

\section{Diurnal temperature range}
\label{sec:DTRmonth}

In the following section, the diurnal temperature range (DTR) is investigated. The CRU TS data is analyzed by month and latitude to determine trends. 

\begin{figure}[ht]
    \centering

    \begin{subfigure}{0.48\textwidth}
        \includegraphics[width=\textwidth]{C:/Users/leonh/Desktop/Praktikum_AWI/NordPol/Lon_66_84/Fit/DTRperMonth.png}
        \caption{DTR}
        \label[type]{subfig:DTRentire}
    \end{subfigure}
    \begin{subfigure}{0.48\textwidth}
        \includegraphics[width=\textwidth]{C:/Users/leonh/Desktop/Praktikum_AWI/NordPol/Lon_66_84/Fit/T_Min_T_Max.png}
        \caption{$T_{\mathrm{min/max}}$}
        \label{subfig:TMinTMax}
    \end{subfigure}

    \begin{subfigure}{0.48\textwidth}
        \centering
        \includegraphics[width = \textwidth]{C:/Users/leonh/Desktop/Praktikum_AWI/NordPol/Lon_66_84/Fit/SNR.png}
        \caption{SNR}
        \label{subfig:SNR}
    \end{subfigure}
    \begin{subfigure}{0.43\textwidth}
        \centering
        \includegraphics[width = \textwidth]{C:/Users/leonh/Desktop/Praktikum_AWI/NordPol/Lon_66_84/ScatterPlot/ScatterTMinTMax.png}
        \caption{$T_{\mathrm{min}}$ vs $T_{\mathrm{max}}$}
        \label{subfig:ScatterMinMax}
    \end{subfigure}

    \caption{Diurnal temperature range change \textbf{(a)} over 50 years for the entire polar region (66-90°). \textbf{(b)} shows, for comparison, the maximum and minimum temperature trends. The variance for these temperatures is much larger compared to \textbf{(a)}. \textbf{(c)} presents the signal-to-noise ratio for the two previous plots.}
    \label{fig:DRTentireSlopes}
\end{figure}

After taking the over all average of the weighted data\footnote{The data are weighted by area. This leads to a stronger emphasis on the "lower" latitudes. The effect of the weighting is decreasingly small at the intervals of 5° chosen later on.},
we investigate the temporal development of DTR monthly average using linear regression. To examine the significance of the trend,
the empirical variance in relation to the linear fit is calculated. 




As show in \cref{fig:DRTentireSlopes}, the changes in DTR over time are significant. But this is also true for the changes in the minimum temperature $T_{\mathrm{min}}$ as for the maximum temperature $T_{\mathrm{max}}$.
Compared to the changes in mean min./max. temperature, the DRT trend does have a bigger SNR for the month april and october, as shown in \cref{subfig:SNR}. This can be explained through the strong correlation between maximum and 
minimum temperature (see \cref{sub@subfig:ScatterMinMax}). Due to this higher SNR, the changes in DTR could be an advantageous indicator for climatic changes in regions with high 
climate variability. Therefore, the robustness of this parameter and trends are investigated in \cref{sec:robustness}.


\section[Robustness]{Robustness of the DTR for subgroups of data}
\label{sec:robustness}


The mathematical derivation of uncertainty is to complex for this internship.
To gain a sense of how reliable the effect of DTR change is, we compare the 
results of the left and right hemisphere  -- the hemisphere from -180 to 0 and 0 to 180 degrees -- and further latitude subsets. 

\begin{figure}[ht]
    \centering
    \begin{subfigure}{0.48\textwidth}
        \centering
        \includegraphics[width = \textwidth]{C:/Users/leonh/Desktop/Praktikum_AWI/NordPolLinks/Lon_66_70/Fit/DTRperMonth.png}
    \end{subfigure}%
    \begin{subfigure}{0.48\textwidth}
        \centering
        \includegraphics[width = \textwidth]{C:/Users/leonh/Desktop/Praktikum_AWI/NordPolRechts/Lon_66_70/Fit/DTRperMonth.png}
    
    \end{subfigure}

    \begin{subfigure}{0.48\textwidth}
        \centering
        \includegraphics[width = \textwidth]{C:/Users/leonh/Desktop/Praktikum_AWI/NordPolLinks/Lon_70_75/Fit/DTRperMonth.png}
    \end{subfigure}%
    \begin{subfigure}{0.48\textwidth}
        \centering
        \includegraphics[width = \textwidth]{C:/Users/leonh/Desktop/Praktikum_AWI/NordPolRechts/Lon_70_75/Fit/DTRperMonth.png}
    \end{subfigure}

    \begin{subfigure}{0.48\textwidth}
        \centering
        \includegraphics[width = \textwidth]{C:/Users/leonh/Desktop/Praktikum_AWI/NordPolLinks/Lon_75_80/Fit/DTRperMonth.png}
    \end{subfigure}%
    \begin{subfigure}{0.48\textwidth}
        \centering
        \includegraphics[width = \textwidth]{C:/Users/leonh/Desktop/Praktikum_AWI/NordPolRechts/Lon_75_80/Fit/DTRperMonth.png}
    \end{subfigure}
    
    \begin{subfigure}{0.48\textwidth}
        \centering
        \includegraphics[width = \textwidth]{C:/Users/leonh/Desktop/Praktikum_AWI/NordPolLinks/Lon_80_84/Fit/DTRperMonth.png}
    \end{subfigure}%
    \begin{subfigure}{0.48\textwidth}
        \centering
        \includegraphics[width = \textwidth]{C:/Users/leonh/Desktop/Praktikum_AWI/NordPolRechts/Lon_80_82/Fit/DTRperMonth.png}
    \end{subfigure}

    \caption{Subfigure with Two Columns}
    \label{fig:DTRleftright}
\end{figure}

% As can be seen in \cref{fig:DTRleftright}, the DTR differs significantly, especially for higher Breitengrade. This may be caused by the 
% selection of data. The data covers the area represented in \cref{fig:datacoverage}. When splitting the data in the 
% left an right hand side, the landmass is not divided equally, but the left half of the earth contains mainland, whereas the right hemisphere contains islands. Therefore the 
% temperatures differ significantly, as on the islands e.g the formation of ice plays a role. 

As depicted in Figure \ref{fig:DTRleftright}, the Diurnal Temperature Range (DTR) exhibits significant variations, particularly at higher
latitudes. This variability may be attributed to the data selection process. The dataset encompasses the geographic region illustrated
in Figure \ref{fig:datacoverage}. When segregating the data into left and right hemispheres, it becomes apparent that the distribution
of landmass is uneven. The left half of the Earth predominantly comprises mainland, whereas the right hemisphere consists mainly of islands.
Consequently, temperature variations are influenced by the geographical circumstances, with islands experiencing unique factors such as ice
formation, which influence their climate differently.

As shown in \cref{app:MaxTemp,app:MinTemp}, the temperatures and their trends differ for the two areas. The mainland dominated left hemisphere experiences 
lower temperatures than the island dominated areas in the right hemisphere.. To further the understanding on the mechanisms driving the DTR we investigate the DTR for two separate stations -- AWIPEV and Neumayer. 




\begin{figure}[h]
    \centering
    \includegraphics[width = 12cm]{C:/Users/leonh/Desktop/Praktikum_AWI/datacoverage.pdf}
    \caption[short]{Data coverage of the CRU data.}
    \label{fig:datacoverage}
\end{figure}
\clearpage

\section{Neumayer III \& AWIPEV}
\label{Sec:LocalDTRTrend}

In the following section, data meterological data from the stations "Neumayer III" and "AWIPEV" is analysed.

% The diurnal temperature trend at AWIPEV, a station in the northern hemisphere near Spitsbergen, shows a decreasing trend. This is in accordance with the observations made in \cref*{sec:DTRmonth}.
% The diurnal cycle of the sun is a subordinate factor in the DTR. In a Fourrier analysis of the temperature data, the diurnal cycle of the sun caused a frequency of 24 hours a signal 
% with the amplitude of 0.3 °C. This account for less than 10 \% of the DTR. The most important contributions to DTR are fluctuations on a shorter time scale and large temperature
% changes on a a multi-day time scale.
\subsection*{Arctic}

The diurnal temperature trend observed at AWIPEV, a research station situated in the northern hemisphere near Spitsbergen, exhibits a declining pattern as shown in \cref{fig:AWIPEVcomparison}. This observation aligns with the findings presented in \cref*{sec:DTRmonth}.
\begin{figure}[ht]
    \centering

    \begin{subfigure}[t]{0.48\textwidth}
        \includegraphics[width=\linewidth]{C:/Users/leonh/Desktop/Praktikum_AWI/Spitzbergen/SB_DTR_year.png}
        \caption{Yearly mean diurnal temperature range at meteorological station AWIPEV near Spitsbergen.}
        \label{fig:DTRyearAWIPEV}
    \end{subfigure}
    \hfill
    \begin{subfigure}[t]{0.48\textwidth}
        \includegraphics[width=\linewidth]{C:/Users/leonh/Desktop/Praktikum_AWI/Spitzbergen/SB_Date_T2Avg.png}
        \caption{Yearly mean temperature at meteorological station AWIPEV near Spitsbergen.}
        \label{fig:TAvgyearAWIPEV}
    \end{subfigure}
    \caption{Comparison of temperature data at AWIPEV station.}
    \label{fig:AWIPEVcomparison}
\end{figure}

% The diurnal cycle of solar radiation, although at mid-latitudes a significant factor, plays a subsidiary role in influencing the Diurnal Temperature Range.
% In a Fourier analysis of the temperature dataset, the diurnal solar cycle produces a signal with a 24-hour frequency and an amplitude of below 0.3 °C.
% However, this contribution accounts for less than 5\% of the overall DTR. The most substantial contributions to DTR are attributable to shorter-term fluctuations and
% significant temperature variations occurring over multiple days.
Although the diurnal cycle of solar radiation plays a subsidiary role in influencing the Diurnal Temperature Range, it is a significant factor at mid-latitudes.
In a Fourier analysis of the temperature dataset, the diurnal solar cycle produces a signal with a 24-hour frequency and an amplitude of less than 0.3 °C. However, this contribution makes up less than 5\% of the overall DTR. Shorter-term fluctuations and significant temperature variations occurring over multiple days are responsible for the most substantial contributions to DTR.

Examining the correlation between the different recorded variables, we can observe a strong connection between the average temperature and the DTR range.

\begin{figure}[ht]
    \centering
    \includegraphics[width = \textwidth]{C:/Users/leonh/Desktop/Praktikum_AWI/Spitzbergen/SB_correlations_month.png}
    \caption{Correlation between average temperature, diurnal temperature range, relative humidity, air pressure, sunshine duration and wind speed.}
    \label{fi:correlationMonth_SB}
\end{figure}

% When plotting the DTR against the average temperature (see \cref{fig:SB_DTR_Month_scatterd}), there seem to be two regimes. One with falling DTR below approximately 
% \SI{0}{\celsius}. For average temperatures above the freezing point the DTR grows again. 
When plotting the DTR against the average temperature (refer to Figure \ref{fig:SB_DTR_Month_scatterd}), two distinct regimes become apparent. In the first regime, the DTR decreases with increasing temperatures for average temperatures up to \SI{2}{\celsius}. However, for average temperatures above this limit, the DTR starts to increase again.

\begin{figure}[ht]
    \centering
    \includegraphics[width = \textwidth]{C:/Users/leonh/Desktop/Praktikum_AWI/Spitzbergen/SB_DTR_T2Avg.png}
    \caption{Comparing Monthly Diurnal Temperature Ranges with Average Monthly Temperatures at Arctic Meteorological Station \textit{AWIPEV} from 1983 to 2022}
    \label{fig:SB_DTR_Month_scatterd}
\end{figure}

% The minimal diurnal temperature difference near the freezing point reveals one dominate
% phenomena. At \SI{0}{\celsius}, the transition of ice from solid to liquid occurs. During this transition, latent heat is absorbed, exerting a cooling effect on the maximum daily temperature. Conversely, temperatures below the freezing point are increased as long there is water in the fluid state present as it releases the latent heat while freezing. IF freezing were the dominant process we would expect the lowest effect at \SI{0}{\celsius}.
% The shift in the DTR minimum can be attributed to the difference between ground temperatures and temperatures at 2 meters above the ground.

Near the freezing point, a minimal diurnal temperature difference reveals one dominant phenomenon: the transition of ice from a solid to a liquid state at \SI{0}{\celsius}. During this transition, latent heat is absorbed, exerting a cooling effect on the maximum daily temperature. Conversely, temperatures below the freezing point increase as long as there is water in the fluid state present, as it releases latent heat while freezing. If freezing were the dominant process and the measurement was ideal, we would expect the lowest effect at \SI{0}{\celsius}.
The shift in the minimum diurnal temperature range could be caused to the disparity between ground temperatures and air temperatures at 2 meters above the ground.

As temperatures rise above freezing point, the snow cover will shrink. This reduction diminishes the insulating properties of snow, rendering the environment more susceptible to external influences. This change in insulation could potentially have a significant impact on diurnal temperature range.


% A significant external influence is an enhanced heat transport mechanism. Altered thermal dynamics, coupled with reduced snow cover, facilitate increased heat exchange within the environment. Consequently, this influences and modulates the daily temperature fluctuation, contributing to the complexity of temperature variations near the freezing point.

\subsection*{Antarctica -- Georg von Neumayer III}

% In the following, the data from Antarctica are analysed in the same way as the AWIPEV data as shown in \cref{fig:GVN_DTR_Month_scatterd}.
% The trends observed at AWIPEV can also be noticed here. However, only the regime for temperatures below 0 degrees Celsius is observable.
In the following section, we analyze the data from Antarctica using the same approach as demonstrated with the AWIPEV data, as illustrated in Figure \ref{fig:GVN_DTR_Month_scatterd}. The trends observed at AWIPEV are also discernible here. However, it's important to note that we can only observe trends in temperatures below 0 degrees Celsius.

\begin{figure}[h!]
    \centering
    \includegraphics[width = \textwidth]{C:/Users/leonh/Desktop/Praktikum_AWI/GVN/GVN_correlation_month.png}
    \caption{Comparing Monthly Diurnal Temperature Ranges with Average Monthly Temperatures at Antarctica's Neumayer Meteorological Station from 1983 to 2022}
    \label{fig:GVN_DTR_Month_scatterd}
\end{figure}


% The strongest correlation observed at Neumayer is between the average temperature and the diurnal temperature range (DTR), which is similar to the observation at AWIPEV.
% The correlation becomes stronger when observing the monthly means compared to the daily means (see \cref{fig:SB_DTR_days_scatterd,fig:GVN_DTR_days_scatterd}).
% This stronger correlation when calculating the monthly mean could be explained by a delay of the effect.
% An example of such an delay could be caused by a lower heat transport in the laminar boundary layer. This delays the warming of the ice until the melting point.

The Neumayer station exhibits a notably robust correlation between average temperature and diurnal temperature range, a pattern that resembles observations at AWIPEV. This correlation strengthens when analyzing monthly means as opposed to daily averages, as depicted in \cref{fig:SB_DTR_days_scatterd,fig:GVN_DTR_days_scatterd}.

The heightened correlation when considering monthly means may be attributed to a delayed effect. One plausible explanation for this delay could be the reduced heat transport within the laminar boundary layer. This delay results in a postponed increase in ground temperature leading to delayed melting. Another explanation could be the melting of snow, which insulates the soil.

\section{Comparing trends in the Arctic and Antarctica}

% The declining trends in DTR between \SI{-20}{\celsius} and \SI{0}{\celsius} mean temperature are comparable at Neumayer and AWIPEV. The signal from Neumayer is noisy in the
% for temperatures close to \SI{0}{\celsius}.The trend of decreasing Diurnal Temperature Range initiates when the extremities of the DTR distribution reach freezing point, under the assumption that the DTR distribution is symmetrically centered around the mean temperature.
The declining trends in DTR between \SI{-20}{\degreeCelsius} and \SI{0}{\degreeCelsius} mean temperature are comparable at Neumayer and AWIPEV. However, the signal from Neumayer exhibits noise, particularly when temperatures are near \SI{0}{\degreeCelsius}. The trend of decreasing Diurnal Temperature Range begins when the outermost values of the DTR distribution approach the freezing point, assuming that the DTR distribution is symmetrically centered around the mean temperature.
\begin{figure}[ht]
    \centering
    \begin{subfigure}[t]{0.49\textwidth}
        \centering
        \includegraphics[width = \textwidth]{C:/Users/leonh/Desktop/Praktikum_AWI/Spitzbergen/SB_DTR_binning.png}
        \caption{AWIPEV}
        \label{subfig:DTRbinningAWIPEV}
    \end{subfigure}
    \hfil
    \begin{subfigure}[t]{0.49\textwidth}
        \centering
        \includegraphics[width = \textwidth]{C:/Users/leonh/Desktop/Praktikum_AWI/GVN/GVN_DTR_binning.png}
        \caption{Neumayer}
        \label{subfig:DTRbinningNeumayer}
    \end{subfigure}
    \caption{Relationship between Average Temperature and Diurnal Temperature Range: The DTR is binned in intervals of \SI{0.5}{\celsius} and averaged }
    \label{fig:DTRbinning}
\end{figure}
This suggests, that the same mechanisms are at work in both situations.  




    % 5.Kapitel Fazit
    %Matteo Kumar - Leonard Schatt
% Fortgeschrittenes Physikalisches Praktikum

% 5. Kapitel Einleitung

\chapter{Summary}
\label{chap:sumamry}



    % Matteo Kumar - Leonard Schatt
% Physikalisches Praktikum

% Anhang

\appendix

% Text

\chapter{Append}
\label{AppChap:Robustness}

\section{Robustness}

\begin{figure}[ht]
    \centering
    \begin{subfigure}{0.48\textwidth}
        \centering
        \includegraphics[width = \textwidth]{C:/Users/leonh/Desktop/Praktikum_AWI/NordPolLinks/Lon_66_70/TMin/TMin_Month_9.png}
        \caption{$T_{min}$ for the left hemisphere between 66 and 70°}
    \end{subfigure}
    \begin{subfigure}{0.48\textwidth}
        \centering
        \includegraphics[width = \textwidth]{C:/Users/leonh/Desktop/Praktikum_AWI/NordPolRechts/Lon_66_70/TMin/TMin_Month_9.png}
        \caption{$T_{min}$ for the right hemisphere between 66 and 70°}
    \end{subfigure}
    
    \begin{subfigure}{0.48\textwidth}
        \centering
        \includegraphics[width = \textwidth]{C:/Users/leonh/Desktop/Praktikum_AWI/NordPolLinks/Lon_70_75/TMin/TMin_Month_9.png}
        \caption{$T_{min}$ between 70 and 75°}
    \end{subfigure}
    \begin{subfigure}{0.48\textwidth}
        \centering
        \includegraphics[width = \textwidth]{C:/Users/leonh/Desktop/Praktikum_AWI/NordPolRechts/Lon_70_75/TMin/TMin_Month_9.png}
        \caption{$T_{min}$ between 70 and 75°}
    \end{subfigure}

        
    \begin{subfigure}{0.48\textwidth}
        \centering
        \includegraphics[width = \textwidth]{C:/Users/leonh/Desktop/Praktikum_AWI/NordPolLinks/Lon_75_80/TMin/TMin_Month_9.png}
        \caption{$T_{min}$ between 75 and 80°}
    \end{subfigure}
    \begin{subfigure}{0.48\textwidth}
        \centering
        \includegraphics[width = \textwidth]{C:/Users/leonh/Desktop/Praktikum_AWI/NordPolRechts/Lon_75_80/TMin/TMin_Month_9.png}
        \caption{$T_{min}$ between 75 and 80°}
    \end{subfigure}

    \begin{subfigure}{0.48\textwidth}
        \centering
        \includegraphics[width = \textwidth]{C:/Users/leonh/Desktop/Praktikum_AWI/NordPolLinks/Lon_80_82/TMin/TMin_Month_9.png}
        \caption{$T_{min}$ between 80 and 82°}
    \end{subfigure}
    \begin{subfigure}{0.48\textwidth}
        \centering
        \includegraphics[width = \textwidth]{C:/Users/leonh/Desktop/Praktikum_AWI/NordPolRechts/Lon_80_82/TMin/TMin_Month_9.png}
        \caption{$T_{min}$ between 80 and 82°}
    \end{subfigure}
    % Include your other subfigures here...
    \caption{Temperature for left and right hemisphere}
    \label{app:MinTemp}
\end{figure}

\begin{figure}[ht]
    \centering
    \begin{subfigure}{0.48\textwidth}
        \centering
        \includegraphics[width = \textwidth]{C:/Users/leonh/Desktop/Praktikum_AWI/NordPolLinks/Lon_66_70/TMax/TMax_Month_9.png}
        \caption{$T_{max}$ for the left hemisphere between 66 and 70°}
    \end{subfigure}
    \begin{subfigure}{0.48\textwidth}
        \centering
        \includegraphics[width = \textwidth]{C:/Users/leonh/Desktop/Praktikum_AWI/NordPolRechts/Lon_66_70/TMax/TMax_Month_9.png}
        \caption{$T_{max}$ for the right hemisphere between 66 and 70°}
    \end{subfigure}
    
    \begin{subfigure}{0.48\textwidth}
        \centering
        \includegraphics[width = \textwidth]{C:/Users/leonh/Desktop/Praktikum_AWI/NordPolLinks/Lon_70_75/TMax/TMax_Month_9.png}
        \caption{$T_{max}$ between 70 and 75°}
    \end{subfigure}
    \begin{subfigure}{0.48\textwidth}
        \centering
        \includegraphics[width = \textwidth]{C:/Users/leonh/Desktop/Praktikum_AWI/NordPolRechts/Lon_70_75/TMax/TMax_Month_9.png}
        \caption{$T_{max}$ between 70 and 75°}
    \end{subfigure}

        
    \begin{subfigure}{0.48\textwidth}
        \centering
        \includegraphics[width = \textwidth]{C:/Users/leonh/Desktop/Praktikum_AWI/NordPolLinks/Lon_75_80/TMax/TMax_Month_9.png}
        \caption{$T_{max}$ between 75 and 80°}
    \end{subfigure}
    \begin{subfigure}{0.48\textwidth}
        \centering
        \includegraphics[width = \textwidth]{C:/Users/leonh/Desktop/Praktikum_AWI/NordPolRechts/Lon_75_80/TMax/TMax_Month_9.png}
        \caption{$T_{max}$ between 75 and 80°}
    \end{subfigure}

    \begin{subfigure}{0.48\textwidth}
        \centering
        \includegraphics[width = \textwidth]{C:/Users/leonh/Desktop/Praktikum_AWI/NordPolLinks/Lon_80_82/TMax/TMax_Month_9.png}
        \caption{$T_{max}$ between 80 and 82°}
    \end{subfigure}
    \begin{subfigure}{0.48\textwidth}
        \centering
        \includegraphics[width = \textwidth]{C:/Users/leonh/Desktop/Praktikum_AWI/NordPolRechts/Lon_80_82/TMax/TMax_Month_9.png}
        \caption{$T_{max}$ between 80 and 82°}
    \end{subfigure}
    % Include your other subfigures here...
    \caption{Temperature for left and right hemisphere}
    \label{app:MaxTemp}
\end{figure}


\clearpage
\section{Local observations}

\begin{figure}[h!]
    \centering
    \includegraphics[width = \textwidth]{C:/Users/leonh/Desktop/Praktikum_AWI/Spitzbergen/SB_correlations_days.png}
    \caption{Comparing daily Diurnal Temperature Ranges with Average Daily Temperatures at Arctic Meteorological Station from AWIPEV from 1983 to 2022}
    \label{fig:SB_DTR_days_scatterd}
\end{figure}\


\begin{figure}[h!]
    \centering
    \includegraphics[width = \textwidth]{C:/Users/leonh/Desktop/Praktikum_AWI/GVN/GVN_correlation_days.png}
    \caption{Comparing daily Diurnal Temperature Ranges with Average Daily Temperatures at Antarctica's Neumayer Meteorological Station from 1983 to 2022}
    \label{fig:GVN_DTR_days_scatterd}
\end{figure}\



    % Literatur
    \bibliographystyle{plain}
    \nocite{*}
    \bibliography{Auswertung.bib}

\end{document}