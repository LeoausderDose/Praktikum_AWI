%Matteo Kumar - Leonard Schatt
% Fortgeschrittenes Physikalisches Praktikum
 
\chapter{Data and methods}
\label{cha:theory}

% \section{Data}
The diurnal temperature range (DTR) data, analyzed at a multi-regional scale in the following section, is derived from the \textit{CRU TS v4.07} dataset \cite{Harris.2020}. This dataset offers a spatial resolution of $\SI{0.5}{\degree} \times \SI{0.5}{\degree}$ and a monthly temporal resolution dating back to 1901.
The diurnal temperature range is defined as the difference
\begin{equation}
    DTR = T_{\mathrm{max}} - T_{\mathrm{min}}
\end{equation}
of the $T_{\mathrm{max}}$ and $T_{\mathrm{min}}$ representing the maximum and minimum temperatures of the day. In spatial averaging we weight by area.

The local data for the Arctic were taken from 'AWIPEV,' the station for continuous meteorological observations in Ny-Ålesund \cite{Maturilli.2020}. For Antarctica, we use the time series data from the 'Georg von Neumayer' station \cite{KonigLanglo.2017}. When multiple datasets for the same quantity exist at the same time, the 
quantities are averaged. 

