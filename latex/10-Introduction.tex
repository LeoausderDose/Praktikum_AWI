%Matteo Kumar - Leonard Schatt
% Fortgeschrittenes Physikalisches Praktikum

% 1. Kapitel Einleitung

\chapter{Introduction}
\label{chap:einleitung}
 
% Studies have show a decrease of the diurnal temperature range (DTR), the difference between daily minimum temperatures ($T_{\mathrm{min}}$) and daily maximum temperatures ($T_{\mathrm{max}}$), on a global scale and in many regions in regional scale. This decrease is caused by the
% mean surface minimum temperatures rising faster then the maximum temperatures, since the 1990s. The reasons for this 
% dynamic is quantitatively not fully understood yet, but the asymmetry in warming was linked to factors of 
% external forcing such as greenhouse gases, aerosols and change of landscape by landuse and secondary effects such as changes in cloudiness and relative humidity. 
% The extend of change in DTR are of magnitudes smaller than the trends in mean surface temperatures ($T_{\mathrm{mean}}$).
% The uncertainty in these trends surpasses that of surface mean temperature, and they were only assessed with medium confidence in the fourth and fifth IPCC reports.



% In this work, we examine the DTR for the arctic region.
% The mean surface temperatures in arctic warms change more rapid than in any other region on earth due to arctic amplification. At the
% same time, the arctic shows great climate variability which decreases the signal-to-noise ratio (SNR) of the trends in mean surface temperature.
% As there is no linear coupling between the diurnal temperature range and the mean temperature, the DTR might be less influenced by the high climate variability.
% In the following, we examine whether the trends in the diurnal temperature range show a better SNR in the 
% arctic region than the mean surface temperature. This could lead to more robust parameter to track changes in the polar regions. 
% % In the sections (sec1), the causes of the change in DTR at the meterological station AWIPEV are more closely investigated.
% % As there are no observable changes due to climate change in mean temperature yet, we investigate in sec (ref), whether the are already changes 
% % in the diurnal temperature range visible. 
% In section (sec1), we investigate factors causing the trends in the Diurnal Temperature Range (DTR) at the AWIPEV meteorological station in Spitzbergen. 
% Since the average temperatures in Antarctica have not changed despite climate change, in section (ref) we address the question of whether there have been shifts in the daily temperature range.
The diurnal temperature range (DTR), the difference between daily minimum temperatures ($T_{\mathrm{min}}$) and daily maximum temperatures ($T_{\mathrm{max}}$), decreases on a global scale and in many regional scales since the 1990s. This decrease is caused by the mean surface minimum temperatures rising faster than the maximum temperatures. The reasons for this dynamic are quantitatively not fully understood yet, but the asymmetry in warming has been linked to factors of external forcing such as greenhouse gases, aerosols, and changes in landscape due to land use, as well as secondary effects such as changes in cloudiness and relative humidity. The extent of change in DTR is of magnitudes smaller than the trends in mean surface temperatures ($T_{\mathrm{mean}}$). The uncertainty in these trends surpasses that of surface mean temperature, and they were only assessed with medium confidence in the fourth and fifth IPCC reports.

In this work, we examine the DTR for the Arctic region. The mean surface temperatures in the Arctic warm more rapidly than in any other region on Earth due to Arctic amplification. At the same time, the Arctic shows great climate variability, which decreases the signal-to-noise ratio (SNR) of the trends in mean surface temperature. As there is no linear coupling between the diurnal temperature range and the mean temperature, the DTR might be less influenced by the high climate variability. In the following, we examine whether the trends in the diurnal temperature range show a better SNR in the Arctic region than the mean surface temperature. This could lead to a more robust parameter to track changes in the polar regions.

In section (sec1), we investigate factors causing the trends in the Diurnal Temperature Range (DTR) at the AWIPEV meteorological station in Spitsbergen. Since the average temperatures in Antarctica have not changed despite climate change, in section (ref), we address the question of whether there have been shifts in the daily temperature range.