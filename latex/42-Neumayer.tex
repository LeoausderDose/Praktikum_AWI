\clearpage

\section{Neumayer III \& AWIPEV}

In the following section, data meterological data from the stations "Neumayer III" and "AWIPEV" is analysed.

% The diurnal temperature trend at AWIPEV, a station in the northern hemisphere near Spitsbergen, shows a decreasing trend. This is in accordance with the observations made in \cref*{sec:DTRmonth}.
% The diurnal cycle of the sun is a subordinate factor in the DTR. In a Fourrier analysis of the temperature data, the diurnal cycle of the sun caused a frequency of 24 hours a signal 
% with the amplitude of 0.3 °C. This account for less than 10 \% of the DTR. The most important contributions to DTR are fluctuations on a shorter time scale and large temperature
% changes on a a multi-day time scale.

The diurnal temperature trend observed at AWIPEV, a research station situated in the northern hemisphere near Spitsbergen, exhibits a declining pattern as shown in \cref{fig:AWIPEVcomparison}. This observation aligns with the findings presented in \cref*{sec:DTRmonth}.
\begin{figure}[ht]
    \centering

    \begin{subfigure}[t]{0.48\textwidth}
        \includegraphics[width=\linewidth]{C:/Users/leonh/Desktop/Praktikum_AWI/Spitzbergen/SB_DTR_year.png}
        \caption{Yearly mean diurnal temperature range at meteorological station AWIPEV near Spitsbergen.}
        \label{fig:DTRyearAWIPEV}
    \end{subfigure}
    \hfill
    \begin{subfigure}[t]{0.48\textwidth}
        \includegraphics[width=\linewidth]{C:/Users/leonh/Desktop/Praktikum_AWI/Spitzbergen/SB_Date_T2Avg.png}
        \caption{Yearly mean temperature at meteorological station AWIPEV near Spitsbergen.}
        \label{fig:TAvgyearAWIPEV}
    \end{subfigure}
    \caption{Comparison of temperature data at AWIPEV station.}
    \label{fig:AWIPEVcomparison}
\end{figure}

The diurnal cycle of solar radiation, although at mid-latitudes a significant factor, plays a subsidiary role in influencing the Diurnal Temperature Range.
In a Fourier analysis of the temperature dataset, the diurnal solar cycle produces a signal with a 24-hour frequency and an amplitude of below 0.3 °C.
However, this contribution accounts for less than 5\% of the overall DTR. The most substantial contributions to DTR are attributable to shorter-term fluctuations and
significant temperature variations occurring over multiple days.

However, if we examine the correlation between the different variables recorded, there seems to be a strong connection between the average temperature and the DTR range.

\begin{figure}[ht]
    \centering
    \includegraphics[width = \textwidth]{C:/Users/leonh/Desktop/Praktikum_AWI/Spitzbergen/SB_correlations_month.png}
    \caption{Correlation between average temperature, diurnal temperature range, relative humidity, air pressure, sunshine duration and wind speed.}
    \label{fi:correlationMonth_SB}
\end{figure}

When plotting the DTR against the average temperature (see \cref{fig:SB_DTR_Month_scatterd}), there seem to be two regimes. One with falling DTR below approximately 
\SI{0}{\celsius}. For average temperatures above the freezing point the DTR grows again. 

\begin{figure}[h!]
    \centering
    \includegraphics[width = \textwidth]{C:/Users/leonh/Desktop/Praktikum_AWI/Spitzbergen/SB_DTR_T2Avg.png}
    \caption[short]{The diurnal temperature range scattered against the average temperature at AWIPEV near Spitsbergen form the years 1983 to 2021}
    \label{fig:SB_DTR_Month_scatterd}
\end{figure}

The minimal diurnal temperature difference near the freezing point reveals one dominate
phenomena. At \SI{0}{\celsius}, the transition of ice from solid to liquid occurs. During this transition, latent heat is absorbed, exerting a cooling effect on the maximum daily temperature. Conversely, temperatures below the freezing point are increased as long there is water in the fluid state present as it releases the latent heat while freezing. IF freezing were the dominant process we would expect the lowest effect at \SI{0}{\celsius}.
The shift in the DTR minimum can be attributed to the difference between ground temperatures and temperatures at 2 meters above the ground.

As temperatures rise above freezing, the snow cover will shrink. This reduction diminishes the insulating properties of snow, rendering the environment more susceptible to external influences. This change in insulation could potentially have a significant impact on diurnal temperature range.


% A significant external influence is an enhanced heat transport mechanism. Altered thermal dynamics, coupled with reduced snow cover, facilitate increased heat exchange within the environment. Consequently, this influences and modulates the daily temperature fluctuation, contributing to the complexity of temperature variations near the freezing point.

\subsection*{Antarctica -- Georg von Neumayer III}

In the following, the data from Antarctica are analysed in the same way as the AWIPEV data.

\begin{figure}[ht]
    \centering
    \includegraphics[options]{}
\end{figure}
